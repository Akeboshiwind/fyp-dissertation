\section{Internet of Things}
``The Internet of things is the network of physical devices, vehicles, home
appliances and other items embedded with electronics, software, sensors,
actuators, and network connectivity which enables these objects to connect and
exchange data \citep{Contributors2018}". This year it is estimated that there
are over 20 billion IoT devices connected to the internet \citep{Statista2015}
with predictions that there will be up to 50 billion devices in 2020
\citep{Evans2011}. With this increasing number of devices inevitably a large
portion of these devices end up in home environments. For these devices to
produce a useful output there needs to be some intelligence managing and
controlling them. This could be the user manually controlling the devices, the
devices communicating with and controlling the other devices in the network or
via a central intelligent controller. The latter is what I will be focusing on.

\section{Home Automation}
In the review of home automation systems done by \cite{Lobaccaro2016}, it can be
observed that most of the systems reviewed are either closed source systems or
do not support a large range of devices. Extensibility is an important feature
of home automation as users won't always want to be locked into using a specific
set of devices when they choose a controller for their home automation system.
An avenue that hasn't been considered is using automated planning to control the
devices, by using a domain-independent or domain configurable device new actions
could be added by the devices as they enter the system, or the controller could
detect new devices and add actions accordingly.

\section{Automated Planning}
The Handbook of Knowledge Representation defines automated planning as "the
deliberation process that chooses and organizes actions by anticipating their
expected effects" \citep{Cimatti2008}.

\section{Classical planning}
Classical planning is an active area of research concerning planning, but the
key point of this area is that it constrains the planners by making a series of
assumptions that limit the types of problems that can be solved by the planners.
These assumptions mainly concern things that usually affect real- world systems
such as implicit time taken or sequential plans. Because of this, most problems
that are likely to be faced when using planning in the real-world are unable to
be solved by a purely classical planner. And as classical planning is the main
focus for much of the research done in planning these assumptions have effects
on other planners developed outside of classical planning.

\section{Domain specific}
This area of planning relies on encoding domain information into the planner
itself, allowing the planner to make efficient plans to solve problems in this
domain. The downside of this type of planner is that it is locked into the
domain of problems it attempts to solve, and cannot be used in other domains
without major work being done on it.

\section{Domain independent}
This is the type of planner that has been the focus of the most research as
classical planning comes under this topic. This style of planning focuses on
making the planning algorithm able to solve problems from different domains,
they do this by having the actions available be part of the input to the
planner. By doing this they are more flexible than domain-specific planners as
they can solve problems for many different domains instead of just one, but
because the domain-specific planner can encode information about the domain,
domain independent planners tend to have worse performance.

\section{Domain configurable} \label{section:domain-configurable}
A middle ground between domain-specific and independent planners is domain
configurable planners, these can solve a wide range of problems well and quickly
as shown in the International Planning Competition in 2000 and 2002, and they do
this without having to program a new core. They do this by allowing domain
information to be encoded in the problem definition allowing the planners to
decrease the search space and thus decrease search time and enabling them to
produce better plans. There are two main types of configurable planning:
hierarchical task network (HTN) planner, which plans by decomposing tasks into
subtasks continually until only primitive tasks remain; and control-rule
planners which define a set of rules for invalid states allowing the planner to
backtrack until a new valid path is found. Benefits of these planners include:
having a mobile core like domain-independent planners which reduces the amount
of work needing to be done to introduce a new problem to the planner; by having
knowledge about the domain, the planner can produce better plans faster.

\section{Comparison}
\cite{Nau2007} compares these different types of planners using three different
comparisons: upfront effort, performance and coverage. In his comparison he
found that configurable planners performed well compared to the other planners,
having lower upfront costs than domain-specific planners but higher than domain
independent, but by they also performed better than domain-independent planners
although worse than domain-specific planners although he did note that "a
sufficiently capable domain-configurable planner should have nearly the same
level of performance because it should be possible to encode the same
domain-specific problem-solving techniques into the domain description". In the
author's research into the International Planning Competition, he found that
domain configurable planners end up having better coverage than the other types,
the author contributes this "partly to efficiency and partly to expressive
power". From this comparison, configurable planners seem to be the best choice,
especially if you are going to be solving different problems like in home
automation.

\section{Conclusion}
The area of home automation is already large and will be growing every year,
because of this it is important that the control systems we develop are easily
extensible and smart enough to solve the increasing demands of the system's
users. Domain configurable planning seems to be a good way of solving this
configurability problem as it can solve problems from many different domains
while still remaining performant. Planning's applicability to this domain has
not been explored before, this might present some problems, but these should
have been explored when applying planning to other domains, so they should be
easy to overcome. Something I have not seen any research on is the effectiveness
of extending a domain via plugins provided by multiple parties, this is
something I plan to base a large part of my project on.

%%% Local Variables:
%%% mode: latex
%%% TeX-master: "../diss"
%%% End: