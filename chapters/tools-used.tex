\section{JSHOP2}
\label{appendix:JSHOP2}
From the SHOP project homepage \citep{UniversityofMaryland2006}:

\begin{quote}
  Our most recent planner is JSHOP2, a Java implementation of SHOP2 (which is
  written in Lisp). In addition to being in Java, JSHOP2 uses a new planner
  compilation technique to synthesize domain-dependent planners from SHOP2
  domain descriptions. This way, JSHOP2 can do a variety of optimizations to
  speed up execution.
\end{quote}

\section{Clojure}
\label{appendix:clojure}
From the Clojure.org homepage \citep{Hickey2018a}:

\begin{quote}
  Clojure is a dynamic, general-purpose programming language, combining the
  approachability and interactive development of a scripting language with an
  efficient and robust infrastructure for multithreaded programming. Clojure is
  a compiled language, yet remains completely dynamic - every feature supported
  by Clojure is supported at runtime. Clojure provides easy access to the Java
  frameworks, with optional type hints and type inference, to ensure that calls
  to Java can avoid reflection.

  Clojure is a dialect of Lisp, and shares with Lisp the code-as-data philosophy
  and a powerful macro system. Clojure is predominantly a functional programming
  language, and features a rich set of immutable, persistent data structures.
  When mutable state is needed, Clojure offers a software transactional memory
  system and reactive Agent system that ensure clean, correct, multithreaded
  designs.
\end{quote}

\section{Leiningen}
\label{appendix:leiningen}
Leiningen is one of the most widely used build tools for clojure. It's use is to
create and manage projects written in the clojure language as well as providing
tools and plugins to aid in development.

\section{Clojure Spec}
\label{appendix:clojure-spec}
From the GitHub page \citep{Hickey2018}:

\begin{quote}
  spec is a Clojure library to describe the structure of data and functions.
  Specs can be used to validate data, conform (destructure) data, explain
  invalid data, generate examples that conform to the specs, and automatically
  use generative testing to test functions.
\end{quote}

\section{Clojure Shell}
\label{appendix:clojure-shell}
Clojure.shell is a library that allows the programmer to pass calls to the
command line of the operating system. This can give them access to external
programs that they would not otherwise be able to access.

%%% Local Variables:
%%% mode: latex
%%% TeX-master: "../diss"
%%% End: