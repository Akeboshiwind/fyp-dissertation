In my work I have described the design of a system that allows users to define
`domain extensions' as a way of combining disparate information about a domain.
This method of combining knowledge provided the advantage of allowing multiple
different third parties to write software that integrated well with the rest of
the system. By allowing multiple third parties to develop software for the
system the software can grow quickly as users discover devices that they wish to
be integrated in the system. Because the software plugins developed by these
third parties is separate to the system I have designed, the core that I have
designed is not effected by the development of these plugins which means that my
system remains easy to maintain no matter the number of devices integrated.

This method also has some disadvantages in that if the third parties are writing
software about similar devices or concepts, then an agreement needs to be made
about the terms used and their meaning in context with the system, if this is
not done then it can be easy for the system to develop complex hard to fix bugs.
A resistance against this could be to have an official `repository' of plugins
that have been compared to a set of standards to reduces the number of such
collisions. This solution introduces extra maintenance costs as well as
possibly stifling the growth of plugins that work differently to how the
standards might expect.

This project has shown that this method is feasible at least in small cases,
more work would need to be done to show it working in larger test cases. Work
has also not been done to address the feedback required for the state of the
world to change nor has work been done to address concerns of actions taking a
real amount of time.

Overall I think this project has successfully implemented a core of a home
automation system. I think more work should be done on the home automation
surrounding my core design and there are many interesting problems still to be
solved surrounding this solution.

%%% Local Variables:
%%% mode: latex
%%% TeX-master: "../diss"
%%% End: