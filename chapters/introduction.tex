In 2010 it was estimated that nearly 12.5 billion devices were connected to the
internet, and it was predicted that by 2020 that number could increase to 50
billion devices \citep{Evans2011}. As the number of internet enabled devices
grows it becomes increasingly likely that these devices will end up in the
average home. This increase in the number of devices and number of interactions
a user will likely have with these devices produces two main problems: how to
effectively manage the devices; and how to allow multiple devices to interact.
These problems can be solved by introducing `control software' which allows the
user to configure how they want the devices in the system to interact with each
other as well as to solve problems and follow any rules they give it.

The most commonly used of such software is software provided by the
manufacturer, that way the user has some guarantee that the software will work
with the devices they own. An issue with this is that the manufacturer has
different goals to the user which can lead to problems such as old devices
ceasing to work on newer software, and devices from other manufacturers not
working with the software at all. Another issue with most device control
software is that it is difficult or impossible to extend to add new or custom
devices with unique behaviors that can interact with the rest of the system
fully.

A solution to this problem could be to have a core system that can be extended
by software written by third parties. This could allow users to develop their
own solutions to devices not working in their system and share their solutions
with other users.

In this document I will show my research into the extent to which current
control software supports extension and investigate the current approaches to
automated planning and how this can help with enabling the extension of home
automation software. I will then design and implement a solution that can
demonstrate how effectiveness automated planning is as a solution to this
problem.

%%% Local Variables:
%%% mode: latex
%%% TeX-master: "../diss"
%%% End: