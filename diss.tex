\documentclass{report}
\usepackage{natbib}
\usepackage{geometry}
\usepackage{pdfpages}
\usepackage{gensymb}
\usepackage[titletoc]{appendix}

\geometry{a4paper, portrait, margin=1in}


\begin{document}

% \includepdf[pages={1}]{data-mining-coversheet.pdf}
% \setcounter{page}{1}

\title{Dissertation} \author{Oliver Marshall}
\maketitle
\newpage

\begin{abstract}
Outline main aims and achievements of project
No more that 300 words
\end{abstract}
\newpage

\renewcommand{\abstractname}{Acknowledgments}
\begin{abstract}
  With many thanks to
  \begin{itemize}
      \item Project Tutor
      \item Peers
      \item Proofreaders
  \end{itemize}
\end{abstract}
\newpage

\tableofcontents
\newpage

\chapter{Introduction}
\section{What is the problem?}
\subsection{Introduce problem}
\subsection{Indicate context for problem}
\section{Why is this problem important/worth investigating?}
*Most important section in this chapter*
\section{The structure of this document}

\chapter{Literature Review}
\section{What research has already been done?}
\subsection{Domain specific vs Domain independent}
\section{Relevance to this work}
\section{Appraisal of other research}
\section{Identify gap in research}

\chapter{Identify Potential Solution}
\section{Use appropriate methods and techniques to:}
\subsection{Identify requirements of the solution}
\subsection{Identify structure and approach of the solution}

\chapter{Requirements Analysis and Requirements Specification}
\section{Process used to capture requirements}
\section{Identify and Discuss Key Requirements}
\subsection{Areas of Difficulty or Conflict}

\chapter{Analyze Work}
\section{Critically analyze all aspects of the project}
\section{Identify possible future work}
\section{Analysis and appraisal of methods and techniques used}
\section{Elaborate on lessons learned}

\chapter{Conclusion}
\begin{itemize}
  \item Should be a natural end point of the argument started by the Introduction
  \item Draws upon problem description and literature review to work done with
    expected results
  \item Should be a natural end point of the argument started by the Introduction
  \item Should draw together lessons learned from critical evaluation
  \item Should clearly identify contributions to current knowledge or practice
\end{itemize}

\begin{appendices}
\chapter{Appendices}
\section{Tools used}
\subsection{Clojure}
Clojure is a dialect of lisp that can either be compiled and run on the JVM or
compiled into javascript to either be run on the browser or in a nodejs VM.
\subsection{Boot}
\section{Guides}
\subsection{User Guide}
\subsection{Management Guide}
\subsection{Developer Guide}
\section{Code}
\subsection{Important Parts Only?}
\end{appendices}

\bibliographystyle{agsm} \bibliography{references}

\end{document}
